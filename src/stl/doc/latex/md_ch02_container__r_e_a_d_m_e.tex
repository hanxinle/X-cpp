

0、vector 扩充存储的方式是指数增加分配空间,并且把自己分配的节点位置。

1、关于 array,荷兰国旗问题,调用 array 和调用 C-\/style 数组的效果是一样的。

2、list 的是双向链表,每个节点包含指向前一个节点和下一个节点的指针,forward\+\_\+list 是指包含指向下个节点的指针的链表,和数据结构课程上学到的一样。list 执行 merge,参数list会清零,合并的 list 大小是排好队的;unique函数会删掉所有重复元素。他们overlie vector(在vector上实现)。

3、deque 是双端队列,和 vector 不同的是它存储的元素不在连续存储位置,比 vector 复杂,在需要处理相对长度大的队列,增加长度更有效率。可以用索引(index)来操作元素,当使用指针指向某个节点时,使用指针增量访问元素会导致未定义的行为。


\begin{DoxyCode}{0}
\DoxyCodeLine{ \{c++\}}
\DoxyCodeLine{deque<int> d \{1, 2, 3, 4\};}
\DoxyCodeLine{d.push\_front(5);}
\DoxyCodeLine{int* i = \&d[0];}
\DoxyCodeLine{}
\DoxyCodeLine{cout << *(i + 2 )  << endl;   //undefined behavior}
\end{DoxyCode}


14、queue 是“覆盖deque的适配器容器”(或者 翻译为在 deque 基础上实现),具有先进先出(\+F\+I\+F\+O)特性;stack 也是覆盖 deque 的适配器容器,后进先出(\+L\+I\+F\+O);priority queue还有一种例外的常规队列,第一个元素是容器内的最大元素(由对排序规范严格执行得到的结果)。元素可以在任何时刻插入,但是只有最大堆元素能被重新读取。同样 overlie vector(在vector上实现)。



5、map 是关联容器,存储成对出现的元素,有四种 map,都靠 key 访问,而不是指针,ordered 指的是基于 key 排序,定制的排序可以在创建 map 时传递给构造函数,mapped 说明一个key和一个value对应,没有value的时候,可能用 N\+U\+LL 表示一个值, 使用 hashing方法便于读取元素,dynamic 动态分配内存,不存在一个固定的size。map容器在根据key访问每个元素一般比 unordered\+\_\+map 效率低/速度慢,但它允许使用基于顺序的直接索引应用在子集上。map 要求独立的 key,插入已经存在的 key 的时候,可能会覆盖掉已经存在的 key 或者被拒绝插入到 map,这取决于插入算法的具体写法(用\mbox{[}\mbox{]},不用 insert 函数)。multimap 就像 regular map,允许使用等价的 keys,不产生异常信息;multimap 不包括\mbox{[}\mbox{]}(),使用 multimap\+::find() 来找到 key;一个 key 对应多个 value,仍可以使用索引,find() 返回第一个,并且这些值是排过序的。inert 函数返回 pair,map 的索引和 key 是否存在于 map 中\mbox{[}iterator itr, boolen inserted\mbox{]}。

6、sets 是非重复元素的集合,元素按照 key 存储,像 map,而不是索引 index,key 就是value,元素插入后不能更改,只能读取或者移除。multimap 允许存在相同 key,它们都有排序和非排序版本。 