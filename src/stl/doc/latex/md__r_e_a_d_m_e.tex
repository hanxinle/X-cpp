这个笔记的前几部分代码来自 Udemy 的课程,总结了一下,发现外语的视频课程学习效率是真的非常低,尤其是这种只有文字没有课件的。

学习 stl 的好教材《\+The C++ programming language-\/4th Edion》,haiyou C++ 标准库。

配合《\+Effective S\+T\+L》学习,并且要留下代码及笔记。笔记是思考成果的凝练,代码是最好的复习资料,直观、实用,光有笔记没有代码的记录时间一长因为遗忘都会成假把式。

\subsection*{1 概述、模板}

stl 以模板技术为基础,其实现、调用有自己的一套逻辑,这个笔记先将 stl 涉及的内容走一遍流程,然后再从资料中总结一些 stl 使用的实用技术。

PS\+: 查阅资料经常出现的几个单词含义,见\mbox{\hyperlink{md_ch01_template__r_e_a_d_m_e}{资料}}。

{\itshape \mbox{\hyperlink{template_8cpp}{template.\+cpp}}} 展示了模板类、模板函数、lambda 模板函数的使用,其中 lambda 模板函数在 Visual Studio 2017 中不可用。模板类中还演示了友元函数 $<$$<$ 写法,用来访问类的私有成员变量。 
\begin{DoxyCode}{0}
\DoxyCodeLine{ \{c++\}}
\DoxyCodeLine{\#include <iostream>}
\DoxyCodeLine{using namespace std;}
\DoxyCodeLine{}
\DoxyCodeLine{template <typename T>}
\DoxyCodeLine{class Container \{}
\DoxyCodeLine{   public:}
\DoxyCodeLine{    explicit Container(T t) : t(t) \{\}}
\DoxyCodeLine{}
\DoxyCodeLine{    friend std::ostream\& operator<<(std::ostream\& os, const Container<T>\& c) \{}
\DoxyCodeLine{        return (os << "private T t == " << c.t);}
\DoxyCodeLine{    \}}
\DoxyCodeLine{}
\DoxyCodeLine{   private:}
\DoxyCodeLine{    T t;}
\DoxyCodeLine{\};}
\DoxyCodeLine{}
\DoxyCodeLine{template <typename T>}
\DoxyCodeLine{bool isGreator(T t1, T t2) \{}
\DoxyCodeLine{    return t1 > t2;}
\DoxyCodeLine{\}}
\DoxyCodeLine{}
\DoxyCodeLine{int main() \{}
\DoxyCodeLine{    Container<int> c\_i\_obj(100);}
\DoxyCodeLine{    Container<string> c\_str\_obj("ssss");}
\DoxyCodeLine{}
\DoxyCodeLine{    cout  << c\_i\_obj << endl;}
\DoxyCodeLine{    cout  << c\_str\_obj << endl;}
\DoxyCodeLine{}
\DoxyCodeLine{    cout << "Test template functions isGreator(template func \& lambda func)"}
\DoxyCodeLine{         << endl;}
\DoxyCodeLine{}
\DoxyCodeLine{    cout << isGreator(110, 22) << endl;       // 1}
\DoxyCodeLine{    cout << isGreator(12.3, 12.2) << endl;    // 1}
\DoxyCodeLine{    cout << isGreator("AAA", "ZZZ") << endl;  // 1}
\DoxyCodeLine{    cout << isGreator('z', 'a') << endl;      // 1}
\DoxyCodeLine{}
\DoxyCodeLine{    // template lambda feature not supported in visual studio 2017}
\DoxyCodeLine{    auto f = []<typename T>(T t1, T t2) \{ return t1 > t2; \};}
\DoxyCodeLine{    cout << "lambda function isGreator(T t1, T t2)" << endl;}
\DoxyCodeLine{    cout << f(110, 22) << endl;       // 1}
\DoxyCodeLine{    cout << f(12.3, 12.2) << endl;    // 1}
\DoxyCodeLine{    cout << f("AAA", "ZZZ") << endl;  // 1}
\DoxyCodeLine{    cout << f('z', 'a') << endl;      // 1}
\DoxyCodeLine{}
\DoxyCodeLine{    return 0;}
\DoxyCodeLine{\}}
\DoxyCodeLine{}
\DoxyCodeLine{// build by - g++ xxx.cpp -o xxx -Wall -Wextra --std=c++17}
\end{DoxyCode}
 {\itshape \mbox{\hyperlink{memory_8cpp}{memory.\+cpp}}} 演示了智能指针 unique\+\_\+ptr 的用法,只能指针是现代 C++ 解决资源泄露问题的一个手段,用 valgrind 检测 memory 程序,可以看到资源泄露情况,new 必须和 delete 对应。 
\begin{DoxyCode}{0}
\DoxyCodeLine{ \{c++\}}
\DoxyCodeLine{\#include <iostream>}
\DoxyCodeLine{\#include <memory>}
\DoxyCodeLine{\#include <string>}
\DoxyCodeLine{using namespace std;}
\DoxyCodeLine{}
\DoxyCodeLine{class Widget \{}
\DoxyCodeLine{   public:}
\DoxyCodeLine{    Widget(string s) : str(s) \{}
\DoxyCodeLine{        cout << "Constructing Widget" << endl;}
\DoxyCodeLine{        cout << "str == " << str << endl;}
\DoxyCodeLine{    \}}
\DoxyCodeLine{}
\DoxyCodeLine{    virtual ~Widget() \{}
\DoxyCodeLine{        cout << "Destroying Widget" << endl;}
\DoxyCodeLine{        cout << "str == " << str << endl;}
\DoxyCodeLine{    \}}
\DoxyCodeLine{}
\DoxyCodeLine{   private:}
\DoxyCodeLine{    string str;}
\DoxyCodeLine{\};}
\DoxyCodeLine{}
\DoxyCodeLine{int main() \{}
\DoxyCodeLine{    Widget* w = new Widget("new");}
\DoxyCodeLine{    unique\_ptr<Widget> uw = make\_unique<Widget>("unique\_ptr");}
\DoxyCodeLine{    // 没有delete,内存不会释放}
\DoxyCodeLine{    // delete w;}
\DoxyCodeLine{    return 0;}
\DoxyCodeLine{\}}
\end{DoxyCode}
 valgrind 测试如下:



其它示例还包括了正则表达式、字符串、异常机制,最后的项目是一个比特币相关项目。 
\begin{DoxyCode}{0}
\DoxyCodeLine{ \{c++\}}
\DoxyCodeLine{\#include <curl/curl.h>}
\DoxyCodeLine{}
\DoxyCodeLine{\#include <array>}
\DoxyCodeLine{\#include <cstdio>}
\DoxyCodeLine{\#include <functional>}
\DoxyCodeLine{\#include <iostream>}
\DoxyCodeLine{\#include <memory>}
\DoxyCodeLine{}
\DoxyCodeLine{\#include "json.hpp"}
\DoxyCodeLine{}
\DoxyCodeLine{typedef std::unique\_ptr<CURL, std::function<void(CURL *)>> CURL\_ptr;}
\DoxyCodeLine{}
\DoxyCodeLine{// CURL * 类型在 C 语言中为 void* 类型,它可以指向任何类型}
\DoxyCodeLine{// 因此没有办法提前获知具体的析构函数,需要定制资源释放模块}
\DoxyCodeLine{class CurlHandle \{}
\DoxyCodeLine{   public:}
\DoxyCodeLine{    CurlHandle() : curlptr(curl\_easy\_init(), deleter) \{}
\DoxyCodeLine{        curl\_global\_init(CURL\_GLOBAL\_ALL);}
\DoxyCodeLine{    \}}
\DoxyCodeLine{}
\DoxyCodeLine{    void setUrl(std::string url) \{}
\DoxyCodeLine{        curl\_easy\_setopt(curlptr.get(), CURLOPT\_URL, url.c\_str());}
\DoxyCodeLine{    \}}
\DoxyCodeLine{}
\DoxyCodeLine{    CURLcode fetch() \{ return curl\_easy\_perform(curlptr.get()); \}}
\DoxyCodeLine{}
\DoxyCodeLine{   private:}
\DoxyCodeLine{    CURL\_ptr curlptr;}
\DoxyCodeLine{    constexpr static auto deleter = [](CURL *c) \{}
\DoxyCodeLine{        curl\_easy\_cleanup(c);}
\DoxyCodeLine{        curl\_global\_cleanup();}
\DoxyCodeLine{    \};}
\DoxyCodeLine{\};}
\DoxyCodeLine{}
\DoxyCodeLine{class Bitcoin \{}
\DoxyCodeLine{   public:}
\DoxyCodeLine{    Bitcoin() : curlhandle(\{\}) \{ curlhandle.setUrl(API\_URL); \}}
\DoxyCodeLine{}
\DoxyCodeLine{    void fetchBitcoinData() \{ curlhandle.fetch(); \}}
\DoxyCodeLine{}
\DoxyCodeLine{   private:}
\DoxyCodeLine{    CurlHandle curlhandle;}
\DoxyCodeLine{    static constexpr const char *API\_URL = "https://blockchain.info/ticker";}
\DoxyCodeLine{\};}
\DoxyCodeLine{}
\DoxyCodeLine{int main() \{}
\DoxyCodeLine{    Bitcoin bitcoin;}
\DoxyCodeLine{    bitcoin.fetchBitcoinData();}
\DoxyCodeLine{    return 0;}
\DoxyCodeLine{\}}
\end{DoxyCode}
 \subsection*{2 容器}

不同于 C 语言的数组,及依赖结构体实现的各种逻辑存储结构,如链表,树等。\+S\+T\+L 的容器直接封装了可包含各种数据类型及其操作的抽象数据结构,用户无需关心资源分配、释放、数据类型管理这类问题。\+S\+T\+L 对数据的操作是有自己的特殊逻辑,与 \+C 语言使用下标、指针不同,stl 使用的是迭代器,迭代器也是属于 stl 的一部分。stl 各类容器有各自的特性,如自动管理空间(扩容、回收),元素自动排序,元素无重复,插入操作复杂度低,访问操作复杂度低......因此,选择恰当的容器对设计程序而言很重要。

容器可以分为序列容器(array、vector),元素根据位置组织,位于连续的内存空间;关联容器(maps、sets、undored\+\_\+maps、unordered\+\_\+sets),元素不再顺序存储,而是借助 keys 获取访问权限,元素可以松散排序(lossely ordered)用于查找算法;适配器容器(stack、queue、priority queue)是对现有容器的封装,并且提供了新的 A\+PI.

这是 stl 容器的\mbox{\hyperlink{md_ch02_container__r_e_a_d_m_e}{详细说明}}。

容器不是独立存在的技术,在使用中除了利用迭代器操作之外,容器经常与 $<$algorithm$>$ 联合使用.



文档内容请参考每个文件夹下 /html/index.html 文件,文档用 doxygen 生成.

\subsection*{3 迭代器}

迭代器可以分为:输入、输出、前向、双向、随机读取、连续迭代器(contiguous C++17),通过 iterator\+\_\+traits 每个迭代器表征特性。

迭代器具有如下特性:复制构造(\+X a, X b(a)),赋值初始化(=),析构性,交换性,数值类特性(如 value\+\_\+type,difference\+\_\+type,reference......),反引用($\ast$it),自增。

迭代器是重要的,因为所有的\+S\+T\+L操作是基于迭代器的,迭代器允许以不显示声明(collction-\/agnostic)的方式遍历顺序元素。迭代器可以用于创建构造器(generator)。

为了有效利用迭代器,迭代器的特性必须定义,包含:difference\+\_\+type 表示两个迭代器差值的类型;value\+\_\+type 反引用迭代器得到这个类型,同时禁止对输出迭代器使用;pointer,reference,iterator\+\_\+category。

\subsubsection*{3.\+1 input iterators}

Input iterator add a few small requirements on top of a base iterator.\+It can read from the pointed-\/to element.\+Input iterator 仅仅适用于单通道算法(single pass algo),一旦自增,之前值的所有拷贝都可能无效,例子:steam iterator 从键盘输入得到字符,迭代器自增,原来的输入的字符已经不见了,不要去访问了。

\subsubsection*{3.\+2 output iterators}

output iterators can used in a sequential output operations,where each element pointed by the iterator is written a value only once, and then the iterator is incremented.

Algorithm requireing output iterators should be single-\/pass output algorithm.


\begin{DoxyItemize}
\item each iterator\textquotesingle{}s position is deferenced, once at most. lvalue derefenceable.\+Must be a class or pointer type.
\item equality and inequality may not be defined for output iterators(\+Not required or not guaranteed to be there)
\end{DoxyItemize}

\subsubsection*{3.\+3 forward iterators}

FI Can be used to acess the sequeue of elements in a range in the direction that goes from its begining towards its end.\+There is a key difference from input iterators that input iterators are only single-\/pass guarantedd,FI must be multi-\/pass guarantedd.\+If a FI satisfies the requirements for an output iterator,it is mutable forward iterator.(如果没有自增操作,则永远指向某个元素)

FI satisfies the input iterators, but don\textquotesingle{}t need to statisfy the output iterators.

\subsubsection*{3.\+4 bidirectional iterators}

与\+F\+I相比,可以执行自减操作。

\subsubsection*{3.\+5 Random-\/access iterators}

Random-\/access iterators can be used to acess elements at an arbitrary offset position(任意偏移位置),与指向的元素相关联,但是提供指针一样的功能(functionality)。功能上是最复杂的迭代器种类。

符合双向迭代器的特性,常数时间消耗的任意数量的偏移,

\subsubsection*{3.\+6 auxiliary(辅助) iterator functions}

The iterator library offers some special functions that can be used universally, regardless of the type of iterator they are used on\+:next,prev,advance,distance

The library also provides special functions to access the iterators defined in containers\+: begin,end,rbegin,rend.\+They also are available with a c prefix for const.

\subsubsection*{3.\+7 iterator adapters(适配器)}

适配器接收(take in)迭代器,并且改变它的部分行为。例如,reverse\+\_\+iterator 接收双向迭代器使其逆向运行;move\+\_\+iterator 接收任意类型的input iterator反引用使其产生右值引用(如同 std\+::move 的使用)。

iterator header defines serveral special iterators for developers to use.\+例如:insert\+\_\+iterator, front\+\_\+insert\+\_\+iterator, back\+\_\+insert\+\_\+iterator, ostream\+\_\+iterator, istream\+\_\+iterator, istreambuf\+\_\+iterator, ostreambuf\+\_\+iterator.

\subsection*{4 算法}

在头文件$<$algorithm$>$ 中,counting、searching、sorting、partitioning and transforming ...

All algorithm operate on range \mbox{[}first,last), ref\+: \href{https://en.cppreference.com/w/cpp/algorithm}{\texttt{ https\+://en.\+cppreference.\+com/w/cpp/algorithm}}.

cpp17 add execution policies for algorithm,包括顺序执行,并行执行等,在$<$execution$>$中定义,编译器支持不完全,仅作了解,to compare what compilers can do can use Compiler Ecplorer :https\+://godbolt.org

\subsubsection*{Sequence(顺序) Algorithm}

算法根据是否会对原始数据进行修改分为两种,一般能够明显区分,不确定的时候,在调用算法前对原始数据进行copy操作。

for\+\_\+each 两者皆可,equal binary predicates,两个操作数。

move 操作,transform the owner of items,当不能用copy的时候,如线程等情况,可用 move.

示例,unique\+\_\+ptr 不能被 copy,用move。

本章要学习学习一下lambda、std\+::function的使用。

\subsection*{5 IO}

路径 img/ 下有对 input stream、output stream、io stream 的原理图。

\subsection*{6 内存}

介绍了几种只能指针,及 allocator 使用、调试,未初始化内存相关内容,allocator 据说会在以后的版本中废弃,所以不深究这里的细节。但其调试等内容,对理解 C++ 及系统相关内容可能有帮助。

\subsection*{7 线程}

介绍了几种 C++ 线程同步机制。 \subsection*{PS}

以后章节的内容不是简单的给几个示例程序就能说明白的,而且我已经入门了那些内容,因此仅仅拷贝一些源文件到对应目录,接口调用是容易的,难的是在工程中实践应用.

这是一个关于 stl 的讲座,总体上一窥 stl 。

105 S\+TL Algorithms in Less Than an Hour

\href{https://www.bilibili.com/video/av46316166?from=search&seid=16850541774524193645}{\texttt{ Video Link}}

说明

这个视频对 S\+TL 算法的学习组织进行说明,视频中提到:即便要掌握这些算法的概况,每个部分所花的时间小于1小时都是很现实的,标题只是一个噱头。

目录(视频)


\begin{DoxyItemize}
\item heap
\begin{DoxyItemize}
\item make\+\_\+heap
\end{DoxyItemize}
\item sort
\item permutation
\item numeric algorithms
\begin{DoxyItemize}
\item inner\+\_\+product
\item adjacent\+\_\+difference
\item sample
\end{DoxyItemize}
\item querying of a propertyqu on 2 ranges
\item search
\item 
\end{DoxyItemize}